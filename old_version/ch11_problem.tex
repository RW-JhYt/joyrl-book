\subsection{关键词}

模仿学习(imitation learning,IL):其讨论我们没有奖励或者无法定义奖励但是有与环境进行交互时怎么进行智能体的学习。这与我们平时处理的问题有些类似,因为通常我们无法从环境中得到明确的奖励。模仿学习又被称为示范学习(learning from demonstration)、学徒学习(apprenticeship learning)以及观察学习(learning by watching)等。

行为克隆(behavior cloning):类似于机器学习中的监督学习,通过收集专家的状态与动作等对应信息,来训练我们的网络。在使用时,输入状态就可以输出对应的动作。

数据集聚合(dataset aggregation):用来应对在行为克隆中专家提供不到数据的情况,其希望收集专家在各种极端状态下的动作。

逆强化学习(inverse reinforcement learning,IRL):逆强化学习先找出奖励函数,再用强化学习找出最优演员。这么做是因为我们没有环境中的奖励,但是有专家的示范,使用逆强化学习,我们可以推断专家是因为何种奖励函数才会采取这些动作。有了奖励函数以后就可以使用一般的强化学习方法找出最优演员。

第三人称视角模仿学习(third person imitation learning):一种把第三人称视角所观察到的经验泛化为第一人称视角的经验的技术。


\subsection{习题}

\kw{11-1} 具体的模仿学习方法有哪些?

\kw{11-2} 行为克隆存在哪些问题呢?对应的解决方法有哪些?

\kw{11-3} 逆强化学习是怎么运行的呢?

\kw{11-4} 逆强化学习方法与生成对抗网络在图像生成中有什么异曲同工之处?

